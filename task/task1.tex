\documentclass[a4paper,12pt]{article}

% Packages
\usepackage[utf8]{inputenc} 
\usepackage[T1]{fontenc} 
\usepackage[margin=2.5cm]{geometry}
\usepackage{graphicx}
\usepackage{enumitem}
\usepackage{hyperref}
\usepackage{svg}

% Metadata
\title{Werkzeuge für das wissenschaftliche Arbeiten}
\author{}
\date{Abgabe: 15.12.2023}

\begin{document}

% Title
\maketitle
\vspace{-2em}
\hrule

% Table of Contents
\section*{Inhaltsverzeichnis}
\begin{enumerate}[leftmargin=1.5cm]
    \item[1.] Projektaufgabe \dotfill Seite 1
    \begin{enumerate}[leftmargin=1cm]
        \item[1.1.] Einleitung \dotfill Seite 1
        \item[1.2.] Aufbau \dotfill Seite 2
        \item[1.3.] Methoden \dotfill Seite 2
    \end{enumerate}
    \item[2.] Abgabe \dotfill Seite 3
\end{enumerate}
\hrule

% Main Content
\section{Projektaufgabe}
In dieser Aufgabe beschäftigen wir uns mit Objektorientierung in Python. Der Fokus liegt auf der Implementierung einer Klasse, dabei nutzen wir insbesondere auch Magic Methods.

\begin{figure}[h!]
    \centering
    \includesvg[width=\textwidth]{./../diagram/classes_files.svg}
    \caption{Darstellung der Klassenbeziehungen.}
    \label{fig:class_relations}
\end{figure}

\subsection{Einleitung}
Ein Datensatz besteht aus mehreren Daten, ein einzelnes Datum wird durch ein Objekt der Klasse \texttt{DataSetItem} repräsentiert. Jedes Datum hat einen Namen (Zeichenkette), eine ID (Zahl) und beliebigen Inhalt.

Nun sollen mehrere Daten, Objekte vom Typ \texttt{DataSetItem}, in einem Datensatz zusammengefasst werden. Sie haben sich schon auf eine Schnittstelle und die benötigten Operationen, die ein Datensatz unterstützen muss, geeinigt. Es gibt eine Klasse \texttt{DataSetInterface}, die die Schnittstelle definiert und Operationen jedes Datensatzes angibt. Bisher fehlt aber noch die Implementierung eines Datensatzes mit allen Operationen.

Implementieren Sie eine Klasse \texttt{DataSet} als eine Unterklasse von \texttt{DataSetInterface}.

\subsection{Aufbau}
Es gibt drei Dateien, \texttt{dataset.py}, \texttt{main.py} und \texttt{implementation.py}. 
\begin{itemize}
    \item In der \texttt{dataset.py} befinden sich die Klassen \texttt{DataSetInterface} und \texttt{DataSetItem}.
    \item In der Datei \texttt{implementation.py} muss die Klasse \texttt{DataSet} implementiert werden.
    \item Die Datei \texttt{main.py} nutzt die Klassen \texttt{DataSet} und \texttt{DataSetItem} aus den jeweiligen Dateien und testet die Schnittstelle und Operationen von \texttt{DataSetInterface}.
\end{itemize}

\subsection{Methoden}
Bei der Klasse \texttt{DataSet} sind insbesondere folgende Methoden zu implementieren. Die genaue Spezifikation finden Sie in der \texttt{dataset.py}:

\begin{itemize}
    \item \texttt{\_\_setitem\_\_(self, name, id\_content)}: Hinzufügen eines Datums mit Name, ID und Inhalt.
    \item \texttt{\_\_iadd\_\_(self, item)}: Hinzufügen eines \texttt{DataSetItem}.
    \item \texttt{\_\_delitem\_\_(self, name)}: Löschen eines Datums auf Basis des Namens. Der Name eines Datums ist ein eindeutiger Schlüssel.
    \item \texttt{\_\_contains\_\_(self, name)}: Prüfung, ob ein Datum mit diesem Namen im Datensatz vorhanden ist.
    \item \texttt{\_\_getitem\_\_(self, name)}: Abrufen des Datums über seinen Namen.
    \item \texttt{\_\_and\_\_(self, dataset)}: Schnittmenge zweier Datensätze bestimmen und als neuen Datensatz zurückgeben.
    \item \texttt{\_\_or\_\_(self, dataset)}: Vereinigungen zweier Datensätze bestimmen und als neuen Datensatz zurückgeben.
    \item \texttt{\_\_iter\_\_(self)}: Iteration über alle Daten des Datensatzes (optional mit Sortierung).
    \item \texttt{filtered\_iterate(self, filter)}: Gefilterte Iteration über einen Datensatz, wobei eine Lambda-Funktion mit den Parametern Name und ID als Filter dient.
    \item \texttt{\_\_len\_\_(self)}: Anzahl der Daten in einem Datensatz abrufen.
\end{itemize}

\section{Abgabe}
Programmieren Sie die Klasse \texttt{DataSet} in der Datei \texttt{implementation.py} zur Lösung der oben beschriebenen Aufgabe. Sie können auch direkt auf Ihrem Computer programmieren. Dazu finden Sie alle drei benötigten Dateien zum Download im Moodle.

Das VPL nutzt denselben Code, wobei die \texttt{main.py} um weitere Testfälle und Überprüfungen erweitert wurde. Diese Überprüfungen dienen dazu, sicherzustellen, dass Sie die richtigen Klassen nutzen.

\vspace{1em}
\hrule
\vspace{0.5em}
\noindent\footnotesize $^*$ Dateien befinden sich im Ordner \texttt{/code/} dieses Git-Repositories.

\end{document}
